\section{Conclusion}

This project explored heart rate prediction through two phases: (1) baseline models on 13,855 Endomondo workouts achieved MAE 13.64 BPM, limited by weak correlation ($r=0.254$); (2) transfer learning on 189 Apple Watch workouts with dense HR sampling improved correlation to $r=0.68$ and achieved \textbf{9.61 BPM validation MAE}, meeting our target.

\textbf{Key Lessons:} (1) Data quality $>$ model complexity—189 high-quality samples outperformed 13,855 noisy samples; (2) Correlation acts as a performance ceiling; (3) Transfer learning enables personalization with minimal data by freezing lower layers.

\subsection{Practical Implications}

Our findings have significant implications for wearable health monitoring. The success of transfer learning with only 189 workouts demonstrates that personalized HR prediction models can be deployed after just 2-3 weeks of user data collection. This is particularly valuable for fitness applications where new users expect immediate personalization.

The correlation discovery ($r=0.254 \rightarrow 0.68$) quantifies the value of sensor quality. For manufacturers, investing in 10-12 HR measurements/minute (vs. sparse sampling) directly translates to 30\% improvement in prediction accuracy. This validates the trend toward continuous optical HR monitoring in modern smartwatches.

\subsection{Methodological Insights}

The two-stage fine-tuning strategy proved critical. Stage 1 (freezing layers 0-2) successfully adapted to high-quality data, while Stage 2 (unfreezing layer 2) caused overfitting. This suggests a general principle: \textbf{for small datasets ($<$200 samples), freeze all but the top layer and output head}. The 189-sample threshold appears sufficient for final-layer adaptation but insufficient for deeper network retraining.

Our feature importance analysis revealed that \texttt{hr\_mean} dominated predictions (importance $>0.5$), while kinematic features (\texttt{speed\_std}, \texttt{speed\_mean}) played secondary roles. This suggests future architectures could benefit from attention mechanisms that dynamically weight physiological history versus current kinematic state.

\subsection{Future Work}

\textbf{Multi-user validation:} Test generalization across diverse fitness levels and age groups using the 191 available GPX workouts.

\textbf{Few-shot learning:} Investigate whether meta-learning approaches can achieve $<10$ BPM with only 10-20 workouts per user.

\textbf{Real-time deployment:} Optimize inference for edge devices (smartwatches) with model quantization and pruning.

\textbf{Causal modeling:} Explore altitude gradient and speed acceleration as predictors to capture physiological lag effects.

The path to sub-5 BPM accuracy lies in denser data from modern wearables, not more complex architectures. Future work should prioritize data acquisition strategies over architectural innovation.
